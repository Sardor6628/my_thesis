This chapter delves into the implications of the comparative analysis of mobile application development using Java, Kotlin, and Dart (Flutter), as presented in the previous results section. The findings underscore significant development practices, code quality, and maintainability disparities across these programming frameworks. The discussion further explores strategic enhancements that could be considered for future research and practical application development.
\subsection{Interpretation of Results}
The analysis revealed that while robust and widely used, Java exhibited the highest number of code smells and lines of code (LOC), which may complicate maintenance and scalability. Kotlin and Dart, however, demonstrated a more efficient coding practice by requiring fewer lines of code and presenting fewer critical issues. This efficiency reduces the potential for bugs and enhances the maintainability of applications. For instance, Kotlin's 48,865 LOC and Dart's 174,044 LOC contrast sharply with Java's 52,901 LOC, reflecting a more concise coding structure in newer languages, potentially leading to better manageability.
\par
The SonarQube analysis of open-source projects further illuminated this point. Java projects displayed a substantial prevalence of minor and major issues, with a significant 65\% being minor but still having a notable 9.5\% critical issues. In contrast, Kotlin, despite a lower total LOC, showed a higher percentage of critical issues (18.9\%), suggesting areas needing stringent quality checks. Dart (Flutter), however, managed a larger codebase with a relatively balanced distribution of code smells, maintaining fewer critical issues (6.1\%), which underscores its capability to handle complex applications with better security and stability frameworks.
\subsection{Recommendations for Future Development Practices}
Given the findings, it is recommended that development teams consider adopting newer programming languages like Kotlin and Dart for their projects, especially when developing complex, scalable applications. These languages offer syntactical efficiencies and have advanced features that can significantly reduce the occurrence of code smells and other quality issues.
Moreover, the study highlights the need for more efficient code quality assessment and application development methodologies. Incorporating machine learning models to predict potential code smells and automate code reviews could significantly enhance the efficiency and effectiveness of the development process. This approach not only saves time but also ensures higher standards of code quality.
\subsection{Proposed Enhancements for Future Research}
To extend the reliability and applicability of the findings from this study, future research should consider the following enhancements:
\begin{itemize}
    \item \textbf{Inclusion of More Programming Languages:} Expanding the analysis to include other emerging languages and frameworks could provide a broader perspective on the optimal tools for various development needs.
    \item \textbf{Integration of Additional Metrics: }Besides LOC and code smells, incorporating metrics such as runtime efficiency, user experience, and energy consumption could provide a more comprehensive understanding of each framework's performance.
    \item \textbf{Larger Sample Size: }Analyzing a broader array of projects from diverse repositories could help validate the current findings across different contexts and applications, potentially revealing new insights into the best practices in mobile application development.
\end{itemize}
\subsection{Reflection on Methodological Approaches}
The methodological approach of using SonarQube for individual projects and a cross-section of open-source projects has proven effective in highlighting differences in code quality across frameworks. However, analysis tools must be continuously updated and customized to keep pace with these languages' evolving features. For instance, the community-developed SonarQube plugin for Flutter was crucial in adapting the analysis tool to new technologies, underscoring the importance of community involvement in research methodologies.