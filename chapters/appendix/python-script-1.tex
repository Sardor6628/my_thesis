%Code---------------------
\definecolor{listingcolor}{gray}{1} % 아직 뭔지 잘 모름
\captionsetup[lstlisting]{format=mylst, labelfont=bf, singlelinecheck=off} % 되도록 건들지마
\counterwithin{lstlisting}{section}
\renewcommand{\thelstlisting}{\arabic{section}.\arabic{lstlisting}}

\lstset{
	language			= python,	%언어
	basicstyle			= \ttfamily\footnotesize,
	showspaces			= false,
	showstringspaces	= false,
	extendedchars		= true,
	upquote				= true,
	backgroundcolor=\color{white},   % choose the background color
	breaklines=true,                 % automatic line breaking only at whitespace
	commentstyle=\color{mygreen},    % comment style
	escapeinside={\%*}{*)},          % if you want to add LaTeX within your code
	keywordstyle=\color{blue},       % keyword style
	stringstyle=\color{mymauve},     % string literal style
}
%-------------------------

\section{Python Script for Project Automation and Configuration} \label{app:a2}

\begin{lstlisting}[
	caption				= Python Scripts for Project Automation and Configuration,	% 여기에는 '_'쓸 거면 '\_'로
	label				= muapp,          % 여기에는 '_'쓸 거면  '_'로
	xleftmargin			= 20pt,
	framexleftmargin	= 20pt,
	tabsize				= 4,
	breaklines			= true,
	breakautoindent		= true,
	postbreak			= \space,
%	backgroundcolor		= \color{listingcolor}, 
	frame		= tb,
	numbers		= left,		% 이 밑으론 줄 번호 설정
	stepnumber	= 1,
	numbersep	= 5pt,
	numberstyle	= \tiny,
	escapeinside = ~~
	]	
    import subprocess
    import os
    import requests
    import base64
    def encode_credentials(username, password):
        credentials = f"{username}:{password}"
        encoded_credentials = base64.b64encode(credentials.encode()).decode()
        return encoded_credentials
    
    
    def create_sonar_project(project_key, project_name, encoded_credentials):
        url = f'http://localhost:9000/api/projects/create?project={project_key}&name={project_name}'
        headers = {
            'Authorization': f'Basic {encoded_credentials}',
            'Content-Type': 'application/x-www-form-urlencoded'
        }
        response = requests.post(url, headers=headers)
        return response
    
    
    def generate_sonar_token(project_key, token_name, encoded_credentials):
        url = f'http://localhost:9000/api/user_tokens/generate?name={token_name}&projectKey={project_key}&type=PROJECT_ANALYSIS_TOKEN'
        headers = {
            'Authorization': f'Basic {encoded_credentials}',
            'Content-Type': 'application/x-www-form-urlencoded'
        }
        response = requests.post(url, headers=headers)
        return response
    
    
    def clone_and_setup_sonar(repo_urls, save_directory, username, password):
        encoded_credentials = encode_credentials(username, password)
        os.makedirs(save_directory, exist_ok=True)
    
        for url in repo_urls:
            repo_name = url.split('/')[-1].replace('.git', '')
            clone_path = os.path.join(save_directory, repo_name)
            config_file_path = os.path.join(clone_path, 'sonarqube_configuration.txt')
            config_file_properties = os.path.join(clone_path, 'sonar-project.properties')
    
    
            # Skip cloning if repo is already cloned and configuration file exists
            if os.path.exists(clone_path) and os.path.exists(config_file_path):
                print(f'Repository "{repo_name}" already cloned and configured. Skipping...')
                continue
    
            # Clone repo if it doesn't exist
            if not os.path.exists(clone_path):
                result = subprocess.run(['git', 'clone', url, clone_path], stdout=subprocess.PIPE, stderr=subprocess.PIPE,
                                        universal_newlines=True)
                if result.returncode != 0:
                    print(f'Failed to clone {repo_name}. Error: {result.stderr}')
                    continue
    
            # Proceed with SonarQube project creation and token generation
            print(f'Cloned {repo_name}')
            project_response = create_sonar_project(repo_name, repo_name, encoded_credentials)
            if project_response.status_code == 200:
                print(f'SonarQube project created for {repo_name}')
                token_response = generate_sonar_token(repo_name, f"{repo_name}_token", encoded_credentials)
                if token_response.status_code == 200 and 'token' in token_response.json():
                    token = token_response.json()['token']
                    with open(config_file_properties, 'w') as properties_file:
                        properties_file.write(f"""
                        #Project identification
                        sonar.projectKey={repo_name}
                        sonar.projectName={repo_name}
                        sonar.projectVersion=1.0
                        
                        # Source code location.
                        # Path is relative to the sonar-project.properties file. Defaults to .
                        # Use commas to specify more than one file/folder.
                        # It is good practice to add pubspec.yaml to the sources as the analyzer
                        # may produce warnings for this file as well.
                        sonar.sources=lib,pubspec.yaml
                        sonar.tests=test
                        
                        # Encoding of the source code. Default is default system encoding.
                        sonar.sourceEncoding=UTF-8
                        """)
                    with open(config_file_path, 'w') as token_file:
                        token_file.write(f"""\n# 
                        #Project identification
                        sonar.projectKey={repo_name}
                        sonar.projectName={repo_name}
                        sonar.projectVersion=1.0
                        
                        # Source code location.
                        # Path is relative to the sonar-project.properties file. Defaults to .
                        # Use commas to specify more than one file/folder.
                        # It is good practice to add pubspec.yaml to the sources as the analyzer
                        # may produce warnings for this file as well.
                        sonar.sources=lib,pubspec.yaml
                        #sonar.tests=test
                        
                        # Encoding of the source code. Default is default system encoding.
                        sonar.sourceEncoding=UTF-8
                        
                        
                        run following:
                        
                    sonar-scanner \
      -Dsonar.projectKey={repo_name} \
      -Dsonar.sources=. \
      -Dsonar.host.url=http://localhost:9000 \
      -Dsonar.login={token}
                        
                                                 """)
    
                    print(f'SonarQube configuration saved to {config_file_path}')
                else:
                    print(f'Failed to generate SonarQube token for {repo_name}.')
            else:
                print(f'Failed to create SonarQube project for {repo_name}.')
    
        print('Finished processing repositories and setting up SonarQube projects.')
    
    
    if __name__ == '__main__':
        repo_urls = [
            #Flutter projects
            "https://github.com/shiosyakeyakini-info/miria.git",
            "https://github.com/hoc081098/hoc081098.git",
            "https://github.com/singerdmx/flutter-quill.git"
            "https://github.com/deckerst/aves.git",
            "https://github.com/clragon/e1547.git",
            "https://github.com/realm/realm-dart",
            "https://github.com/KhalidWar/anonaddy",
            "https://github.com/prateekmedia/pstube.git"
        ]
        save_directory = 'flutter_repositories'
        username = 'admin'  # SonarQube username
        password = 'superadmin'  # SonarQube password
        clone_and_setup_sonar(repo_urls, save_directory, username, password)
    
\end{lstlisting}