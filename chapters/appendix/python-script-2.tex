\section{Python Script for Data Retrieval and Report Generation} \label{app:a3}

%-------------------------
\begin{lstlisting}[
	caption				= Python Script for Data Retrieval and Report Generation
	label				= muapp,          % 여기에는 '_'쓸 거면  '_'로
	xleftmargin			= 20pt,
	framexleftmargin	= 20pt,
	tabsize				= 4,
	breaklines			= true,
	breakautoindent		= true,
	postbreak			= \space,
%	backgroundcolor		= \color{listingcolor}, 
	frame		= tb,
	numbers		= left,		% 이 밑으론 줄 번호 설정
	stepnumber	= 1,
	numbersep	= 5pt,
	numberstyle	= \tiny,
	escapeinside = ~~
	]	
import requests
import pandas as pd

pd.set_option('display.max_rows', None)  
pd.set_option('display.max_columns', None) 
pd.set_option('display.width', 1000)  
pd.set_option('display.max_colwidth', None)

def fetch_project_metrics_and_issues_summary(base_url, projects, types="CODE_SMELL", ps=500):
    project_summaries = []  

    for project in projects:
        # Fetch issues summary
        issues_params = {
            "componentKeys": project,
            "types": types,
            "ps": ps
        }

        issues_response = requests.get(f"{base_url}/api/issues/search", params=issues_params)

        # Fetch project metrics
        metrics_params = {
            "component": project,
            "metricKeys": "ncloc,complexity,violations"
        }

        metrics_response = requests.get(f"{base_url}/api/measures/component", params=metrics_params)

        if issues_response.status_code == 200 and metrics_response.status_code == 200:
            issues_data = issues_response.json()
            metrics_data = metrics_response.json()

            # Parse issues data
            issues = issues_data.get("issues", [])
            severity_counts = {
                "BLOCKER": 0,
                "CRITICAL": 0,
                "MAJOR": 0,
                "MINOR": 0,
                "INFO": 0
            }
            for issue in issues:
                if issue["severity"] in severity_counts:
                    severity_counts[issue["severity"]] += 1

            # Parse metrics data for ncloc (Lines of Code)
            measures = metrics_data["component"]["measures"]
            loc = next((measure["value"] for measure in measures if measure["metric"] == "ncloc"), "0")

            # Prepare the summary for the current project
            project_summary = {
                "NAME": project,
                "BLOCKER": severity_counts["BLOCKER"],
                "CRITICAL": severity_counts["CRITICAL"],
                "MAJOR": severity_counts["MAJOR"],
                "MINOR": severity_counts["MINOR"],
                "INFO": severity_counts["INFO"],
                "total": len(issues),
                "LOC": loc  # Add LOC to the project summary
            }

            project_summaries.append(project_summary)
        else:
            print(
                f"Failed to fetch data for project {project}, issues status code: {issues_response.status_code}, metrics status code: {metrics_response.status_code}")

    return pd.DataFrame(project_summaries)


base_url = "http://localhost:9000"
projects = ["Kanban_board_flutter","day_night_time_picker", "android-tv-app", "ConsumerFlutterApp", "flutter-chat-craft",
            "Flutter-TDD-Clean-Architecture-E-Commerce-App", "flutter_samples", "Kanban_board_flutter", "flutter-quill",
            "Musify", "openfoodfacts-dart", "Personal-Finance-Manager",
            "mobile", "spotube", "anonaddy", "pstube"]
project_metrics_and_issues_summary_df = fetch_project_metrics_and_issues_summary(base_url, projects)
# Display the table
print(project_metrics_and_issues_summary_df)

# Save the DataFrame to a CSV file
report_path = "flutter-report.csv"
project_metrics_and_issues_summary_df.to_csv(report_path, index=False)
print(f"Flutter Report saved to {report_path}")

projects = ["Kanban_board_java", 'CountdownTimer2', '2048-android2', 'Easy-Attendance-App', 'StopWatchRemade',
            'Post-it-Notes-App', 'VideoPlayer', 'CurrencyConverter', 'passwordgenerator', 'finance-manager',
            'todo-list', 'pain-diary', 'qr-scanner', 'tape-measure', 'pedometer', 'weather']

project_metrics_and_issues_summary_df = fetch_project_metrics_and_issues_summary(base_url, projects)
# Display the table
print(project_metrics_and_issues_summary_df)

# Save the DataFrame to a CSV file
report_path = "java-report.csv"
project_metrics_and_issues_summary_df.to_csv(report_path, index=False)
print(f"Java Report saved to {report_path}")

projects = ["Kanban_board_kotlin", "vocable-android", "plees-tracker", "NotyKT",
            "muzei", "awaker", "GitExplorer-Android",
            'pdf-viewer-pro', 'vocable', 'MyNotes',
            'RecurringExpenseTracker',
            'Simple-Dialer', 'PhoneBook',
            'SoundMeter1', 'Simple-Draw', 'unitconverter', 'Simple-Voice-Recorder']

project_metrics_and_issues_summary_df = fetch_project_metrics_and_issues_summary(base_url, projects)
# Display the table
print(project_metrics_and_issues_summary_df)

# Save the DataFrame to a CSV file
report_path = "kotlin-report.csv"
project_metrics_and_issues_summary_df.to_csv(report_path, index=False)
print(f"Kotlin Report saved to {report_path}")

\end{lstlisting}