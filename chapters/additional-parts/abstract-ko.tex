
\begin{eabstract}
    모바일 앱 개발 방법에 대해 많은 접근법이 제안되었지만, 개발자들은 적합한 방법을 선택하는 데 어려움을 겪고 있습니다. 이 연구는 주로 Java에서 Kotlin으로의 선호도 변화와 Flutter의 사용 증가에 중점을 두고, 네이티브 및 크로스플랫폼 애플리케이션 개발 접근 방식을 비교합니다. 이 연구는 Java, Kotlin, Dart(Flutter)를 사용하여 동일한 애플리케이션을 생성함으로써 모바일 애플리케이션 개발에서 개발자의 프로그래밍 언어 및 프레임워크 선택에 영향을 미치는 요인에 대한 실용적인 통찰력을 제공합니다. 또한, 이 연구는 45개의 오픈소스 GitHub 저장소에서 코드 품질을 검토함으로써 개발의 모범 사례를 탐구합니다. 연구는 반자동 SonarQube 평가를 사용하여 LOC 및 코드 스멜을 평가함으로써 특정 언어나 프레임워크 선택이 코드 유지 관리 및 개발 효율성에 미치는 영향을 결정합니다. 예비 결과는 두 접근 방식에서 생성된 코드 품질의 차이를 보여주며, 개발자가 코드 스멜을 줄이고 프로젝트 유지 관리를 개선하기 위해 언어와 프레임워크 선택을 최적화하는 데 도움이 되는 정보를 제공합니다.
\end{eabstract}