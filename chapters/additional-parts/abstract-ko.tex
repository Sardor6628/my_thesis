
\begin{eabstract}
    모바일 앱 개발 방법에 대해 많은 접근법이 제안되었지만, 개발자들은 적합한 방법 선택에 어려움을 겪고 있다. 이 연구는 네이티브 및 크로스 플랫폼 어플리케이션 개발 접근 방식을 비교하며, 특히 Java에서 Kotlin으로의 선호도 변화와 Flutter의 사용 증가에 중점을 두고 비교한다. 이 연구는 Java, Kotlin, Dart(Flutter)를 사용하여 같은 어플리케이션을 작성함으로써, 모바일 어플리케이션 개발에서 프로그래밍 언어 및 프레임워크 선택에 영향을 미치는 요인에 관한 실질적인 통찰을 제공한다. 또한, 이 연구는 45개의 GitHub 공개 소스 프로그램의 코드 품질도 검사함으로써 모범 개발 사례를 모색한다. 구체적으로, 반자동 SonarQube 평가를 사용하여 LOC 및 코드 스멜을 평가하고, 이를 통해 특정 언어나 프레임워크 선택에 따라 코드의 유지보수 및 개발 효율성에 미치는 영향을 탐구한다. 실험 결과, 작성된 코드 품질의 차이를 알 수 있었다. 본 연구는 코드 스멜을 줄이고 프로젝트 유지보수를 개선하는 데 적합한 프로그래밍 언어와 프레임워크를 선택하는 데 도움이 될 것으로 기대된다.
\end{eabstract}