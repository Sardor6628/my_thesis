\thispagestyle{empty}
\begin{center}
    \vskip 3.5cm
    {\Large{Comparative Analysis of Cross-Platform and Native Mobile App Development Approaches}} \\
    \vskip 15mm
    {\large{Ibrokhimov Sardorbek Rustam Ugli}} \\
    \vskip 10mm
    {Department of Information Convergence Engineering} \\
    {The Graduate School} \\
    {Pusan National University} \\
    \vskip 15mm
    {\large{Abstract}}\\
    \vskip 10 mm
\end{center}
Though many approaches to developing mobile applications have been suggested up to now, developers have difficulties selecting the right one. This study compares native and cross-platform application development approaches, particularly focusing on the shift in preference from Java to Kotlin and the increasing use of Flutter. This research offers practical insights into factors influencing developers’ choice of programming languages and frameworks in mobile application development by creating identical applications using Java, Kotlin, and Dart (Flutter). Furthermore, this study explores the best practices for development by examining the quality of code in 45 open-source GitHub repositories. The study evaluates LOC and code smells using semi-automated SonarQube assessments, including the measurement of severity levels, to determine the effects of selecting a specific language or framework on code maintainability and development efficiency. Preliminary findings show differences in the quality of the code produced by the two approaches, offering developers useful information on how to best optimize language and framework selection to reduce code smells and improve project maintainability.