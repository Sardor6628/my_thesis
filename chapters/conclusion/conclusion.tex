This thesis has provided a comprehensive comparative analysis of mobile application development across three primary programming languages: Java, Kotlin, and Dart (Flutter). Through the detailed examination of each language's performance in developing a Kanban board application and the subsequent analysis of open-source projects using SonarQube, significant insights into the coding efficiencies, maintainability, and overall code quality have been gleaned. The findings underscore each language's inherent strengths and weaknesses, offering a nuanced understanding of their suitability for various development contexts.
\par
Remarkably, the analysis of the Kanban board application development and the SonarQube assessments of open-source projects revealed that Kotlin and Dart exhibited superior maintainability and code quality compared to Java. Kotlin had a significantly lower lines of code (LOC) at 1,467 for the Kanban board and 48,865 for open-source projects, with fewer code smells (24 in the Kanban board and 647 in open-source projects) and a higher proportion of critical issues. Dart (Flutter) demonstrated an ability to manage larger codebases effectively, with 1,515 LOC for the Kanban board and 174,044 LOC for open-source projects, and a lower proportion of critical code smells (22 in the Kanban board and 1,303 in open-source projects). Java, despite its robustness, showed the highest number of code smells, indicating potential challenges in maintenance and scalability, with 1,748 LOC for the Kanban board and 52,901 LOC for open-source projects, and 62 code smells in the Kanban board and 3,590 code smells in open-source projects.
\par
When selecting a programming language, it is crucial to consider project requirements, time scope, and developer experience. The language should be chosen based on the project's specific needs, such as the intended platform, complexity of the app, and feature requirements. The timeline for development and deployment must also be considered, and languages that allow for fast development and deployment might be preferable in time-sensitive projects. Finally, the development team's experience and familiarity with a particular language can significantly impact the efficiency and quality of the development process. Therefore, selection should be made considering the team's proficiency in a given language, making the transition easier.
\par
The thesis findings suggest that developers should adopt newer programming languages like Kotlin and Dart to enhance their coding practices, as they have the potential to increase efficiency and reduce code complexity. Additionally, it is recommended that they utilize advanced tools for code analysis and quality assurance, such as SonarQube, to identify and address code smells and other quality issues regularly. Continuous learning and training are also essential for developers to keep up-to-date with the latest developments in programming languages and frameworks, which can help improve the maintainability of their applications.
\par
Future research should explore the integration of additional programming languages and newer frameworks to update our understanding of best practices in application development continually. Moreover, adopting more comprehensive metrics, including runtime efficiency and user experience, could provide a broader perspective on the performance of different languages. Extending the analysis to include a larger dataset of open-source projects could also enhance the generalizability of the findings.
\par
This thesis highlights the importance of carefully selecting a mobile application development language based on specific project requirements, the time scope for development, and the development team’s experience. While newer languages like Kotlin and frameworks like Flutter offer substantial benefits in terms of efficiency and maintainability, the choice of technology should always be tailored to the project's specific needs and the development team's capabilities. Through thoughtful selection and implementation of development languages, developers can optimize their processes, enhance the quality of their applications, and ultimately, contribute to the advancement of mobile technology. This research advocates for a balanced approach to framework selection, emphasizing continuous quality assurance and adopting best practices as pivotal to the long-term success of mobile application development projects.
